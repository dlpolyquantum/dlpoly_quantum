
\section{Constructing \D{}}

\subsection{Overview}

The \D{} executable program is constructed as follows.

\begin{enumerate}
\item \D{} is supplied as a gzipped tar file. This must be unpacked
to create the \D{} directory (section \ref{directory}).
\item In the {\em build} subdirectory you will find the required \D{}
makefile (see section \ref{compile} and Appendix \ref{A1}, where a
sample Makefile is listed). This must
be copied into the subdirectory containing the relevant source
code. In most cases this will be the {\em source} subdirectory.
\item The makefile is executed with the appropriate keywords
(section \ref{compile}) which selects for specific computers (including
serail and parallel machines) and the appropriate communication software.  
\item The makefile produces the executable version of the code, which
as a default will be named DLPOLY.X and located in the {\em execute}
subdirectory.
\item DL\_POLY also has a Java GUI. The files for
this are stored in the subdirectory {\em java}. Compilation of this is
simple and requires running the javac compiler and the jar
utility. Details for these procedures are provided in the GUI manual
\cite{smith-gui}.
\item To run the executable for the first time you require (as a
minimum) the files CONTROL, FIELD and CONFIG (and possibly TABLE or
TABEAM if you have tabulated potentials). These must be present in
the directory from which the program is executed. (See section
\ref{inputfiles} for the description of the input files.)
\item Executing the program will most often produce the files OUTPUT,
REVCON and REVIVE (and optionally STATIS, HISTORY, RDFDAT and
ZDNDAT) in the executing directory.  (See section \ref{outputfiles}
for the description of the output files.)
\end{enumerate}

This simple procedure is enough to create a standard version to run
most \D{} applications. However it sometimes happens that additional
modifications may be necessary. 

On starting, \D{} scans the input data and makes an estimate of the
sizes of the arrays it requires to do the simulation. Sometimes the
estimates are not good enough. The most common occurrences of this are
NPT and NST simulations, or simulations where the local density on the
MD cell may significantly exceed the mean density of the cell (systems
with a vacuum gap for example). Under these circumstances arrays
initally allocated may be insufficent. In which case \D{} may report a
memory problem and request that you recompile the code with
hand-adjusted array dimensions. This topic is dealt with more fully in
Appendix C.

