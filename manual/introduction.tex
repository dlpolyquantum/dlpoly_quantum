
\section{The DL\_POLY Classic Package}

\D{} \cite{smith-96a} is a molecular simulation package designed to
facilitate molecular dynamics simulations of macromolecules, polymers,
ionic systems, solutions and other molecular systems on a distributed
memory parallel computer. The package was originally written to
support the UK project CCP5\index{CCP5} by Bill Smith and Tim Forester
\cite{smith-87a} under grants from the Engineering and Physical
Sciences Research Council and is the copyright of the Science and
Technology Facilities Council (STFC).

\D{} is based on a replicated data parallelism. It is suitable for simulations
of up to 30,000 atoms on up to 100 processors.  Though it is designed for
distributed memory parallel machines, we have taken care to ensure that it
can, with minimum modification, be run on common workstations. Scaling up
a simulation from a small workstation to a massively parallel machine is
therefore a useful feature of the package.

We request that our users respect the copyright of the \D{} source and not alter
any authorship or copyright notices within. 

Further information about the \D{} package can be obtained from
the CCP5 Program Library website: \[\SOFT{}.\index{WWW}\]

\section{Functionality}

The following is a list of the features in \D{}.

\subsection{Molecular Systems}

\D{} will simulate the following molecular species:

\begin{enumerate}
\item Simple atomic systems and mixtures e.g. Ne, Ar, Kr, etc.
\item Simple unpolarisable point ions e.g. NaCl, KCl, etc.
\item Polarisable point ions and molecules e.g. MgO, H$_{2}$O etc.D
\item Simple rigid molecules\index{rigid body} e.g. CCl$_{4}$, SF$_{6}$, Benzene, etc.
\item Rigid molecular\index{rigid body} ions with point charges e.g. KNO$_{3}$,
(NH$_{4}$)$_{2}$SO$_{4}$, etc.
\item Polymers with rigid bonds\index{constraints!bond} e.g. C$_{n}$H$_{2n+2}$
\item Polymers with rigid bonds\index{constraints!bond} and point charges e.g. proteins
\item Macromolecules and biological systems
\item Molecules with flexible bonds
\item Silicate glasses and zeolites
\item Simple metals and alloys e.g. Al, Ni, Cu etc.
\item Covalent systems e.g. C, Si, Ge, SiC, SiGe etc.
\end{enumerate}

In addition, since version 1.10, \DD{} can perform a path integral
molecular dynamics (PIMD) simulation of atomistic systems and systems
composed of flexible molecules.

\subsection{The \D{} Force Field\index{force field}}

The \D{} force field\index{force field}\index{force field!DL\_POLY}
includes the following features: \index{potential!electrostatic}

\begin{enumerate}
\item All common forms of non-bonded\index{potential!nonbonded} atom-atom potential;
\item Atom-atom (site-site) Coulombic\index{potential!electrostatic} potentials;
\item Valence angle\index{potential!valence angle} potentials;
\item Dihedral\index{potential!dihedral} angle potentials;
\item Inversion\index{potential!inversion} potentials;
\item Improper dihedral\index{potential!improper dihedral} angle potentials;
\item 3-body\index{potential!three-body} valence angle\index{potential!valence angle} and hydrogen bond\index{potential!bond} potentials;
\item 4-body\index{potential!four-body} inversion\index{potential!inversion} potentials;
\item Finnis-Sinclair and embedded atom type density dependent potentials\index{potential!metal} (for metals) \cite{finnis-84a,johnson-89a}.
\item The Tersoff density dependent potential for covalent systems \cite{tersoff-89a}.
\end{enumerate}

The parameters describing many of these these potentials may be
obtained, for example, from the GROMOS\index{force
  field!GROMOS}\index{GROMOS} \cite{gunsteren-87a},
Dreiding\index{force field!Dreiding} \cite{mayo-90a} or
AMBER\index{force field!AMBER}\index{AMBER} \cite{weiner-86a}
force field, which share functional forms.

Note that \D{} does not have its own \index{force field} force
field. However, the force fields mentioned above can be adapted for
\D{} using the program \DLF{}, which was developed at Daresbury
Laboratory by Chin Yong, specifically for constructing \DD{}
compatible force field files. \DLF{} can be found via the CCP5 Program
Library website \SOFT{}. Otherwise adaptation by hand is possible for
simple systems.

\subsection{Boundary Conditions}

\D{} will accommodate the following boundary conditions\index{boundary conditions}:

\begin{enumerate}
\item None e.g. isolated polymer in space.
\item Cubic periodic boundaries\index{boundary conditions}.
\item Orthorhombic periodic boundaries.
\item Parallelepiped periodic boundaries.
\item Truncated octahedral periodic boundaries.
\item Rhombic dodecahedral periodic boundaries.
\item Slab (x,y periodic, z nonperiodic).
\item Hexagonal prism periodic boundaries.
\end{enumerate}

These are describe in detail in Appendix \ref{A5}.

\subsection{The Java Graphical User Interface}
\index{Graphical User Interface}

\D{} has a Graphical User Interface (GUI) written specifically for the package
in the Java programming language from Sun microsystems.  The Java programming
environment is free and it is particularly suitable for building graphical
user interfaces. An attractive aspect of java is the portability of the
compiled GUI, which may be run without recompiling on any Java supported
machine. The GUI is an integral component of the \D{} package and is available
under the same terms. (See \cite{smith-gui}.)

Other graphics packages are suitable for use with \D. In particular
VMD \cite{VMD}\index{VMD} and Aten \cite{aten} \index{Aten} are
recommended. VMD is particularly good for visualizing \DD{} HISTORY
files and Aten is useful for constructing systems for simulation.

\subsection{Algorithms\index{algorithm}}

\subsubsection{Parallel Algorithms\index{parallelisation}}

\D{} exclusively employs the {\bf Replicated
  Data}\index{parallelisation!Replicated Data} parallelisation strategy
\cite{smith-94a,smith-94b} (see section \ref{parallelisation}).

\subsubsection{Molecular Dynamics Algorithms}

The \D{} MD algorithms\index{algorithm} are optionally available in the
form of the Verlet Leapfrog\index{algorithm!Verlet leapfrog} or the
Velocity Verlet\index{algorithm!velocity Verlet} integration
algorithms\index{algorithm} \cite{allen-89a}.

In the leapfrog scheme a parallel version of the SHAKE
algorithm\index{algorithm!SHAKE} \cite{ryckaert-77a,smith-94b} is used
for bond constraints\index{constraints!bond} and a similar adaptation
of the RATTLE\index{algorithm!RATTLE} algorithm \cite{andersen-83a} is
implmented in the velocity Verlet scheme.

Rigid body rotational motion \index{rigid body} is handled under the
leapfrog scheme with Fincham's implicit quaternion\index{quaternions}
algorithm\index{algorithm!FIQA} (FIQA) \cite{fincham-92a}.  For
velocity Verlet integration of rigid bodies \D{} uses the `NOSQUISH'
algorithm of Miller {\em et al} \index{algorithm!NOSQUISH} \cite{miller-02a}.

Rigid molecular species linked by rigid bonds\index{rigid body} are
handled with an algorithm of our own devising, called the QSHAKE
algorithm\index{algorithm!QSHAKE} \cite{forester-96a} which has been
adapted for both leapfrog and velocity Verlet schemes.

NVE, NVT, NPT and N\mat{\sigma}T ensembles\index{ensemble} are
available, with a selection of thermostats\index{thermostat} and
barostats\index{barostat}. The velocity Verlet versions are based on
the reversible integrators of Martyna {\em et al} \cite{martyna-96a}. 

The NVT algorithms in \D{} are those of Evans\index{ensemble!Evans
NVT} \cite{evans-84a}, Berendsen\index{ensemble!Berendsen NVT}
\cite{berendsen-84a}; and Hoover\index{ensemble!Hoover NVT}
\cite{hoover-85a}. The NPT algorithms are those of
Berendsen\index{ensemble!Berendsen NPT} \cite{berendsen-84a} and
Hoover\index{ensemble!Hoover NPT} \cite{hoover-85a} and the N$\sigma$T
algorithms are those of Berendsen\index{ensemble!Berendsen N$\sigma$T}
\cite{berendsen-84a} and Hoover\index{ensemble!Hoover N$\sigma$T}
\cite{hoover-85a}.

The full range of MD algorithms\index{algorithm} available in \D{} is
described in section \ref{integration}.

\subsubsection{Structure Relaxation Algorithms}

\D{} has a selection of structure relaxation methods available. These 
are useful to improve the starting structure of a molecular dynamics 
simulation. The algorithms available are:

\begin{enumerate}
\item `Zero' temperature molecular dynamics (sometimes called damped molecular dynamics)\index{minimisation!zero temperature};
\item Conjugate gradients minimisation\index{minimisation!conjugate gradients};
\item `Programmed' energy minimisation, involving both molecular dynamics and conjugate gradients \index{minimisation!programmed}.
\end{enumerate}

Starting structure minimisation is described in section \ref{minimisation}.

\section{Programming Style}

The programming style of \D{} is intended to be as uniform as
possible. The following stylistic rules apply throughout. Potential
contributors of code are requested to note the stylistic convention.

\subsection{Programming Language}

\D{} is written exclusively in FORTRAN 90\index{FORTRAN 90}. Use is made
of F90 Modules. Explicit type declaration is used throughout.

\subsection{Memory Management}

In \D{}, the major array dimensions are calculated at
the start of execution and the associated arrays created through the
dynamic array allocation features of FORTRAN 90\index{FORTRAN 90}.

\subsection{Target Computers}

\D{} is intended for distributed memory parallel computers.
However, versions of the program for serial computers are easily
produced. To facilitate this all machine specific calls are located in
dedicated FORTRAN\index{FORTRAN 90} routines, to permit substitution
by appropriate alternatives.

\D{} will run on a wide selection of computers. This includes most single
processor workstations for which it requires a FORTRAN 90 compiler and
(preferably) a UNIX environment. It has also been compiled for a Windows PC
using both the GFORTRAN and G95 FORTRAN compiler augmented by the CygWin UNIX
shell.  The Message Passing Interface (MPI) software is essential for parallel
execution.

\subsection{Version Control System (CVS)}

\D{} was developed with the aid of the CVS\index{CVS} version control
system. We strongly recommend that users of \D{} adopt this (or a
similar) system for local development of the \D{} code, particularly
where several users access the same source code. References to
suitable software can be found on the internet.

\subsection{Required Program Libraries}

\D{} is, for the most part, self contained and does not require access
to additional program libraries. The exception is the MPI software
library required for parallel execution.  

Users requiring the Smoothed Particle Mesh Ewald (SPME)
\index{Ewald!SPME} method may prefer to use
a proprietary 3D FFT other than the one ({\sc dlpfft3}) supplied with
the package for optimal performance. There are comments in the source
code which provide guidance for applications on Cray and IBM
computers, which use the routines {\sc ccfft3d} and {\sc dcft3}
respectively. Similarly users will find comments for the public domain
FFT routine {\sc fftwnd\_fft}.

\subsection{Internal Documentation}

All subroutines are supplied with a header block of
FORTRAN\index{FORTRAN 90} COMMENT
records giving:

\begin{enumerate}
\item The name of the author and/or modifying author
\item The version number or date of production
\item A brief description of the function of the subroutine
\item A copyright statement
\end{enumerate}

Elsewhere FORTRAN\index{FORTRAN 90} COMMENT cards are used liberally.

\subsection{Subroutine/Function Calling Sequences}

The variables in the subroutine arguments are specified in the order:

\begin{enumerate}
\item logical and logical arrays
\item character and character arrays
\item integer
\item real and complex
\item integer arrays
\item real and complex arrays
\end{enumerate}

This is admittedly arbitrary, but it really does help with error detection.

\subsection{FORTRAN Parameters}

All global parameters defined by the FORTRAN\index{FORTRAN 90}
parameter statements are specified in the module: {\sc
setup\_module}. All parameters specified in {\sc setup\_module}
are described by one or more comment cards.

\subsection{Arithmetic Precision}

All real variables and parameters are specified in 64-bit precision
(i.e real*8).

\subsection{Units}
\label{units}
Internally all \D{} subroutines and functions assume the use of the
following defined {\em molecular units}\index{units!DL\_POLY}:

\begin{enumerate}

\item The unit of time ($t_{o}$) is $1\times 10^{-12}$ seconds (i.e. picoseconds).
\item The unit of length ($\ell_{o}$) is $1\times 10^{-10}$ metres
(i.e. \AA ngstroms).
\item  The unit of mass ($m_{o}$) is $1.6605402\times 10^{-27}$
kilograms (i.e. atomic mass units).
\item The unit of charge ($q_{o}$) is $1.60217733\times 10^{-19}$
coulombs (i.e. unit of proton charge).
\item The unit of energy ($E_{o}=m_{o}(\ell_{o}/t_{o})^{2}$) is 
$1.6605402\times 10^{-23}$~Joules (10~J~mol$^{-1}$).
\item The unit of pressure\index{units!pressure} (${\cal P}_{o}=E_{o}\ell_{o}^{-3}$) is 
$1.6605402\times 10^{7}$~Pascal ($163.882576$~atm).
\item Planck's constant ($\hbar$) which is $6.350780719\times E_{o}t_{o}$.
\end{enumerate}

\noindent In addition the following conversion factors are used:

\noindent The coulombic\index{potential!electrostatic} conversion factor ($\gamma_{o}$) is:
\[\gamma_{o}=\frac{1}{E_{o}} \left [
\frac{q_{o}^{2}}{4\pi\epsilon_{o}\ell_{o}}\right ]= 138935.4835 \]
such that:
\[U_{MKS}=E_{o}\gamma_{o}U_{Internal}\]
Where $U$ represents the configuration energy.

\noindent The Boltzmann factor ($k_{B}$) is $0.831451115~E_{o}K^{-1}$,
such that:
\[T=E_{kin}/k_{B}\]
represents the conversion from kinetic energy (in internal units) to
temperature (in Kelvin).

Note: In the \D{} CONTROL and OUTPUT files, the
pressure\index{units!pressure} is given in units of kilo-atmospheres
(k-atm) at all times. The unit of energy is either \D{} units specified above,
or in other units specified by the user at run time. The default is
DL\_POLY units.

\subsection{Error Messages}

All errors detected by \D{} during run time initiate a call
to the subroutine {\sc error}, which prints an error message in the
standard output file and terminates the program.  All
terminations of the program are global (i.e. every node of the
parallel computer will be informed of the termination condition and
stop executing.)

In addition to terminal error messages, \D{} will sometimes
print warning messages. These indicate that the code has detected
something that is unusual or inconsistent. The detection is non-fatal,
but the user should make sure that the warning does represent a
harmless condition.

More on error handling can be found in section (\ref{errorprocess}).

\section{The \D{} Directory Structure}
\label{directory}

The entire \D{} package is stored in a Unix directory structure.  The
topmost directory is named {\em dl\_poly\_class}.
Beneath this directory are several sub-directories:
\begin{tabbing}
XXXXXXXXXXXXXXX\=XXXXXXXX\= \kill
sub-directory\> ~ \> contents \\
\> \> \\
{\em source} \> primary subroutines for the \D{} package\index{sub-directory!source}\\
{\em utility} \> subroutines, programs and example data for all utilities\index{sub-directory!utility} \\
{\em data} \> example input and output files for \D{} \index{sub-directory!data}\\
{\em execute} \> the \D{} run-time directory \index{sub-directory!execute}\\
{\em build} \> makefiles to assemble and compile \D{} programs\index{sub-directory!build} \\
{\em java} \> directory of Java and FORTRAN routines for the Java GUI \index{sub-directory!java}\\
\end{tabbing}

A more detailed description of each sub-directory follows.

\subsection{The {\em source} Sub-directory}

In this sub-directory all the essential source code for \D{}, excluding the
utility software. The modules are assembled at compile time using an
appropriate makefile.

\subsection{The {\em utility} Sub-directory}

This sub-directory stores all the utility programs in \D.  Users who devise
their own utilities are advised to store them in the {\em utility}
sub-directory.

\subsection{The {\em data} Sub-directory}

This sub-directory contains examples of input and output files for
testing the released version of \D. The examples of input data
are copied into the {\em execute} sub-directory when a program is being
tested. The test cases are documented in chapter \ref{data}.

\subsection{The {\em execute} Sub-directory}

In the supplied version of \D{}, this sub-directory contains only
a few macros for copying and storing data from and to the {\em data}
sub-directory and for submitting programs for execution. (These are
decribed in section \ref{macros}.) However when
a \D{} program is assembled using its makefile, it will be placed
in this sub-directory and will subsequently be executed from here. The
output from the job will also appear here, so users will find it
convenient to use this sub-directory if they wish to use \D{} as
intended. (The experienced user is not absolutely required to use
\D{} this way however.)

\subsection{The {\em build} Sub-directory}

This sub-directory contains the standard makefiles for the creation
(i.e. compilation and linking) of the \D{} simulation programs.
The makefiles supplied select the appropriate subroutines from the
{\em source} sub-directory and deposit the executable program in the
{\em execute} directory. The user is advised to copy the appropriate
makefile into the {\em source} directory, in case any modifications are
required. The copy in the {\em build} sub-directory will then serve as a
backup. 

\subsection{The {\em java} Sub-directory}

The \D{} Java Graphical User Interface (GUI)\index{Graphical User
  Interface} is based on the Java language developed by Sun.  The Java
source code for this GUI is to be found in this sub-directory togther
with a java executable GUI.jar which can be run immediately if a Java
runner is available on your system.

\[java -jar GUI.jar \]

Otherwise the source files are complete and sufficient to create a
working GUI, provided the user has a java compiler installed (or the
Java Development Kit, 1.4 or above), both of which are available free
from \[\href{http://www.oracle.com}{http://www.oracle.com}\] 
The GUI, once compiled, may be executed on any machine where Java is installed.
(See \cite{smith-gui}.)

\section{Obtaining the Source Code}

\index{FORGE} \D{} source code and the associated test data is available from
CCPForge at \[ \FORGE. \]


\section{Other Information}

The CCP5 Program Library website: \[\SOFT{} \index{WWW}\] provides
information on other Daresbury Laboratory programs related to \DD{}.

\clearpage
